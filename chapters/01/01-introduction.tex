ইন্টিগ্রেশনের ধারণা, ক্যালকুলাসের একটি মৌলিক হাতিয়ার, বক্ররেখার অধীনে এলাকা খুঁজে বের করতে বা অ্যান্টিডেরিভেটিভ গননা করার ক্ষেত্রে এর ঐতিহ্যগত ভূমিকার বাইরে প্রসারিত।
আমাদের (দ্বাদশ শ্রেণির গণিত ছাত্রদের) ইন্টিগ্রেলের প্রয়োগের কৌতুহলপূর্ণ ক্ষেত্রের সাথে পরিচয় করিয়ে দেয়, যেখানে বিভিন্ন শাখায় বাস্তব-বিশ্বের সমস্যা সমাধানের জন্য একীকরণের শক্তি ব্যবহার করা হয়।

এই ক্লাসে অন্বেষণ করা প্রাথমিক অ্যাপ্লিকেশনগুলির মধ্যে একটি হল এলাকা এবং আয়তনের গণনা।
বক্ররেখাগুলিকে ছেদ করে বা ওভারল্যাপ করে এমন পরিস্থিতি সহ, বক্ররেখার মধ্যে ক্ষেত্র খুঁজে বের করতে হয় তার পূর্ণাঙ্গ ব্যবহার শিখি। এই জ্ঞানটি ত্রিমাত্রিক স্থানগুলিতে প্রসারিত, যেখানে একটি পৃষ্ঠের নীচের আয়তন অবিকল কৌশল ব্যবহার করে সঠিকভাবে গণনা করা যেতে পারে।

জ্যামিতির বাইরে, পূর্ণাঙ্গ পদার্থবিদ্যায় গুরুত্বপূর্ণ ভূমিকা পালন করে। পরিমাণের পরিবর্তনের হার বোঝার মাধ্যমে, শিক্ষার্থীরা গতি সম্পর্কিত সমস্যা সমাধানের জন্য পূর্ণাঙ্গ প্রয়োগ করতে পারে, যেমন একটি বস্তু দ্বারা ভ্রমণ করা দূরত্ব, তার বেগ এবং ত্বরণ নির্ণয় করা।
এই অ্যাপ্লিকেশনগুলি বিমূর্ত গাণিতিক ধারণা এবং ভৌত জগতের মধ্যে ব্যবধানকে সেতু করে, বাস্তব-বিশ্বের ঘটনা বিশ্লেষণ এবং ব্যাখ্যা করার জন্য একটি শক্তিশালী টুলসেট প্রদান করে।

অর্থনীতি এবং জীববিজ্ঞানে, ইন্টিগ্রেলের প্রয়োগ গতিশীল সিস্টেমের মডেলিং এবং বিশ্লেষণে সহায়ক। এটি জনসংখ্যা বৃদ্ধির পূর্বাভাস, সম্পদের প্রবাহ বোঝা, বা সময়ের সাথে পরিবর্তনশীল পরিবর্তনের মূল্যায়ন করা হোক না কেন, একীকরণের নীতিগুলি এই ক্ষেত্রগুলিতে বিভিন্ন অ্যাপ্লিকেশন খুঁজে পায়।
\newpage
