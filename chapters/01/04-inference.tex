উপসংহারে, দ্বাদশ শ্রেণির গণিতে সমাকলনের প্রয়োগের অধ্যয়ন এমন একটি জগতের প্রবেশদ্বার খুলে দেয় যেখানে বিমূর্ত গাণিতিক ধারণাগুলি বিভিন্ন ক্ষেত্রে ব্যবহারিক প্রাসঙ্গিকতা খুঁজে পায়।
যখন তারা বক্ররেখার অধীনে এলাকা গণনা করার জটিলতার মধ্য দিয়ে নেভিগেট করে, পদার্থবিদ্যায় গতির ভবিষ্যদ্বাণী করে এবং অর্থনীতি ও জীববিজ্ঞানে গতিশীল সিস্টেমের মডেলিং করে, শিক্ষার্থীরা একটি ব্যাপক দক্ষতা অর্জন করে যা ঐতিহ্যগত গাণিতিক শিক্ষার সীমানা অতিক্রম করে।
এইভাবে সমাকলনের প্রয়োগ একটি সেতু হিসাবে আবির্ভূত হয় তাত্ত্বিক জ্ঞানকে তার বাস্তব প্রকাশের সাথে সংযুক্ত করে, শিক্ষাগত অভিজ্ঞতাকে সমৃদ্ধ করে এবং বিশ্ব সম্পর্কে আমাদের বোঝার উপর গাণিতিক নীতির ব্যাপক প্রভাবকে হাইলাইট করে।

\newpage
